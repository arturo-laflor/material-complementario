\documentclass[]{book}
\usepackage{lmodern}
\usepackage{amssymb,amsmath}
\usepackage{ifxetex,ifluatex}
\usepackage{fixltx2e} % provides \textsubscript
\ifnum 0\ifxetex 1\fi\ifluatex 1\fi=0 % if pdftex
  \usepackage[T1]{fontenc}
  \usepackage[utf8]{inputenc}
\else % if luatex or xelatex
  \ifxetex
    \usepackage{mathspec}
  \else
    \usepackage{fontspec}
  \fi
  \defaultfontfeatures{Ligatures=TeX,Scale=MatchLowercase}
\fi
% use upquote if available, for straight quotes in verbatim environments
\IfFileExists{upquote.sty}{\usepackage{upquote}}{}
% use microtype if available
\IfFileExists{microtype.sty}{%
\usepackage{microtype}
\UseMicrotypeSet[protrusion]{basicmath} % disable protrusion for tt fonts
}{}
\usepackage[margin=1in]{geometry}
\usepackage{hyperref}
\hypersetup{unicode=true,
            pdftitle={Sleep quality analysis},
            pdfauthor={Arturo Laflor},
            pdfborder={0 0 0},
            breaklinks=true}
\urlstyle{same}  % don't use monospace font for urls
\usepackage{natbib}
\bibliographystyle{apalike}
\usepackage{color}
\usepackage{fancyvrb}
\newcommand{\VerbBar}{|}
\newcommand{\VERB}{\Verb[commandchars=\\\{\}]}
\DefineVerbatimEnvironment{Highlighting}{Verbatim}{commandchars=\\\{\}}
% Add ',fontsize=\small' for more characters per line
\usepackage{framed}
\definecolor{shadecolor}{RGB}{248,248,248}
\newenvironment{Shaded}{\begin{snugshade}}{\end{snugshade}}
\newcommand{\KeywordTok}[1]{\textcolor[rgb]{0.13,0.29,0.53}{\textbf{{#1}}}}
\newcommand{\DataTypeTok}[1]{\textcolor[rgb]{0.13,0.29,0.53}{{#1}}}
\newcommand{\DecValTok}[1]{\textcolor[rgb]{0.00,0.00,0.81}{{#1}}}
\newcommand{\BaseNTok}[1]{\textcolor[rgb]{0.00,0.00,0.81}{{#1}}}
\newcommand{\FloatTok}[1]{\textcolor[rgb]{0.00,0.00,0.81}{{#1}}}
\newcommand{\ConstantTok}[1]{\textcolor[rgb]{0.00,0.00,0.00}{{#1}}}
\newcommand{\CharTok}[1]{\textcolor[rgb]{0.31,0.60,0.02}{{#1}}}
\newcommand{\SpecialCharTok}[1]{\textcolor[rgb]{0.00,0.00,0.00}{{#1}}}
\newcommand{\StringTok}[1]{\textcolor[rgb]{0.31,0.60,0.02}{{#1}}}
\newcommand{\VerbatimStringTok}[1]{\textcolor[rgb]{0.31,0.60,0.02}{{#1}}}
\newcommand{\SpecialStringTok}[1]{\textcolor[rgb]{0.31,0.60,0.02}{{#1}}}
\newcommand{\ImportTok}[1]{{#1}}
\newcommand{\CommentTok}[1]{\textcolor[rgb]{0.56,0.35,0.01}{\textit{{#1}}}}
\newcommand{\DocumentationTok}[1]{\textcolor[rgb]{0.56,0.35,0.01}{\textbf{\textit{{#1}}}}}
\newcommand{\AnnotationTok}[1]{\textcolor[rgb]{0.56,0.35,0.01}{\textbf{\textit{{#1}}}}}
\newcommand{\CommentVarTok}[1]{\textcolor[rgb]{0.56,0.35,0.01}{\textbf{\textit{{#1}}}}}
\newcommand{\OtherTok}[1]{\textcolor[rgb]{0.56,0.35,0.01}{{#1}}}
\newcommand{\FunctionTok}[1]{\textcolor[rgb]{0.00,0.00,0.00}{{#1}}}
\newcommand{\VariableTok}[1]{\textcolor[rgb]{0.00,0.00,0.00}{{#1}}}
\newcommand{\ControlFlowTok}[1]{\textcolor[rgb]{0.13,0.29,0.53}{\textbf{{#1}}}}
\newcommand{\OperatorTok}[1]{\textcolor[rgb]{0.81,0.36,0.00}{\textbf{{#1}}}}
\newcommand{\BuiltInTok}[1]{{#1}}
\newcommand{\ExtensionTok}[1]{{#1}}
\newcommand{\PreprocessorTok}[1]{\textcolor[rgb]{0.56,0.35,0.01}{\textit{{#1}}}}
\newcommand{\AttributeTok}[1]{\textcolor[rgb]{0.77,0.63,0.00}{{#1}}}
\newcommand{\RegionMarkerTok}[1]{{#1}}
\newcommand{\InformationTok}[1]{\textcolor[rgb]{0.56,0.35,0.01}{\textbf{\textit{{#1}}}}}
\newcommand{\WarningTok}[1]{\textcolor[rgb]{0.56,0.35,0.01}{\textbf{\textit{{#1}}}}}
\newcommand{\AlertTok}[1]{\textcolor[rgb]{0.94,0.16,0.16}{{#1}}}
\newcommand{\ErrorTok}[1]{\textcolor[rgb]{0.64,0.00,0.00}{\textbf{{#1}}}}
\newcommand{\NormalTok}[1]{{#1}}
\usepackage{longtable,booktabs}
\usepackage{graphicx,grffile}
\makeatletter
\def\maxwidth{\ifdim\Gin@nat@width>\linewidth\linewidth\else\Gin@nat@width\fi}
\def\maxheight{\ifdim\Gin@nat@height>\textheight\textheight\else\Gin@nat@height\fi}
\makeatother
% Scale images if necessary, so that they will not overflow the page
% margins by default, and it is still possible to overwrite the defaults
% using explicit options in \includegraphics[width, height, ...]{}
\setkeys{Gin}{width=\maxwidth,height=\maxheight,keepaspectratio}
\IfFileExists{parskip.sty}{%
\usepackage{parskip}
}{% else
\setlength{\parindent}{0pt}
\setlength{\parskip}{6pt plus 2pt minus 1pt}
}
\setlength{\emergencystretch}{3em}  % prevent overfull lines
\providecommand{\tightlist}{%
  \setlength{\itemsep}{0pt}\setlength{\parskip}{0pt}}
\setcounter{secnumdepth}{5}
% Redefines (sub)paragraphs to behave more like sections
\ifx\paragraph\undefined\else
\let\oldparagraph\paragraph
\renewcommand{\paragraph}[1]{\oldparagraph{#1}\mbox{}}
\fi
\ifx\subparagraph\undefined\else
\let\oldsubparagraph\subparagraph
\renewcommand{\subparagraph}[1]{\oldsubparagraph{#1}\mbox{}}
\fi

%%% Use protect on footnotes to avoid problems with footnotes in titles
\let\rmarkdownfootnote\footnote%
\def\footnote{\protect\rmarkdownfootnote}

%%% Change title format to be more compact
\usepackage{titling}

% Create subtitle command for use in maketitle
\newcommand{\subtitle}[1]{
  \posttitle{
    \begin{center}\large#1\end{center}
    }
}

\setlength{\droptitle}{-2em}
  \title{Sleep quality analysis}
  \pretitle{\vspace{\droptitle}\centering\huge}
  \posttitle{\par}
  \author{Arturo Laflor}
  \preauthor{\centering\large\emph}
  \postauthor{\par}
  \predate{\centering\large\emph}
  \postdate{\par}
  \date{2017-07-27}

\usepackage{booktabs}
\usepackage{amsthm}
\usepackage{multirow}
\usepackage{multicol}
\usepackage{float}
\usepackage{tabularx}
\makeatletter
\def\thm@space@setup{%
  \thm@preskip=8pt plus 2pt minus 4pt
  \thm@postskip=\thm@preskip
}
\makeatother

\begin{document}
\maketitle

{
\setcounter{tocdepth}{1}
\tableofcontents
}
\chapter{Description of the work}\label{description-of-the-work}

This work describes the progress that has been made so far in estimating
sleep quality, based on sleep hygiene factors. The characterization of
the phenomenon is being carried out in two stages, the first, described
in this work, has to do with the generation of a model that estimates
the quality of sleep and that has been trained with subjective data from
a questionnaire applied to 341 volunteers . The second stage is in
process, this stage includes a logbook that records the factors of a
person's sleep hygiene every day and the measurement of quality is not
subjective, but is measured by an electronic device. In this way, in the
end, an analysis of both characterizations will be made and a model will
be constructed that takes into account the two perspectives to obtain a
better approximation in their predictions.

\chapter{Data preparation}\label{dataprep}

\begin{Shaded}
\begin{Highlighting}[]
\KeywordTok{library}\NormalTok{(ggplot2)}
\KeywordTok{library}\NormalTok{(bitops)}
\KeywordTok{library}\NormalTok{(RCurl)}
\CommentTok{#funciones online}
\NormalTok{script <-}\StringTok{ }\KeywordTok{getURL}\NormalTok{(}\StringTok{"https://raw.githubusercontent.com/arturo-laflor/util-R-codes/master/QOfCategoricalF.R"}\NormalTok{, }\DataTypeTok{ssl.verifypeer =} \OtherTok{FALSE}\NormalTok{)}
\KeywordTok{eval}\NormalTok{(}\KeywordTok{parse}\NormalTok{(}\DataTypeTok{text =} \NormalTok{script),}\DataTypeTok{envir=}\NormalTok{.GlobalEnv)}

\NormalTok{script <-}\StringTok{ }\KeywordTok{getURL}\NormalTok{(}\StringTok{"https://raw.githubusercontent.com/arturo-laflor/util-R-codes/master/QOfContinuousF.R"}\NormalTok{, }\DataTypeTok{ssl.verifypeer =} \OtherTok{FALSE}\NormalTok{)}
\KeywordTok{eval}\NormalTok{(}\KeywordTok{parse}\NormalTok{(}\DataTypeTok{text =} \NormalTok{script),}\DataTypeTok{envir=}\NormalTok{.GlobalEnv)}

\NormalTok{script <-}\StringTok{ }\KeywordTok{getURL}\NormalTok{(}\StringTok{"https://raw.githubusercontent.com/arturo-laflor/util-R-codes/master/multiplot.R"}\NormalTok{, }\DataTypeTok{ssl.verifypeer =} \OtherTok{FALSE}\NormalTok{)}
\KeywordTok{eval}\NormalTok{(}\KeywordTok{parse}\NormalTok{(}\DataTypeTok{text =} \NormalTok{script),}\DataTypeTok{envir=}\NormalTok{.GlobalEnv)}

\NormalTok{knitr::opts_chunk$}\KeywordTok{set}\NormalTok{(}\DataTypeTok{echo =} \OtherTok{TRUE}\NormalTok{,}\DataTypeTok{root.dir=}\StringTok{"C:/Master/Libro-Pearson-CETYS/"}\NormalTok{,}\DataTypeTok{fig.pos =} \StringTok{'H'}\NormalTok{)}
\end{Highlighting}
\end{Shaded}

Hasta aquí ya sabes que los datos son de suma importancia para tomar
decisiones que pueden hacer crecer tu capacidad de ingresos en un
negocio. Cómo ya se te mencionó, dichos datos normalmente se encontrarán
almacenados en el disco duro de alguna computadora destinada para este
fin (cuando las empresas son grandes, a las computadoras que almacenan
los datos se conocen como ``servidores de base de datos''). Sin embargo,
antes de que los datos estén almacenados y listos para ser utilizados en
el análisis que permitirá extraer información útil para el negocio, los
datos normalmente deben recibir cierto tratamiento. Este proceso se
denomina ``preparación de datos'' porque verifica que los datos con los
que se realizarán los análisis para la toma de decisiones estén
``libres'' de imperfecciones hasta donde sea posible. De esta manera, se
puede decir que los datos ``se preparan'' previamente para el análisis,
de tal forma que los algoritmos (programas de computadora especializado
en tareas que generan información relevante), trabajen con datos cuya
calidad les permita un desempeño óptimo. Las principales imperfecciones
de los datos son: Registros con datos faltantes, registros con datos en
formato incorrecto, registros con valores atípicos y variables con datos
irrelevantes. Estas imperfecciones en los datos mayormente se deben a
omisiones de datos al llenar los campos de un formulario manual o
electrónico, errores humanos en la captura y programas sin validaciones
rigurosas de entrada que eviten su ingreso.

El proceso de preparación de los datos inicia con una exploración visual
de los datos que permite identificar algunos problemas que son evidentes
y que se pueden solucionar, antes de que los procesos automatizados
entren en acción. Por ejemplo, en una exploración visual de un conjunto
de datos, es posible identificar si existen registros donde ninguno de
los campos tenga datos, así mismo, podría advertirse de alguna variable
donde todos los datos se encuentren almacenados en un formato no
conveniente para hacer análisis. Las acciones a tomar en estos casos
particulares, pueden ser: Eliminar los registros sin datos y hacer un
proceso de transformación de formato a uno que convenga para el
análisis. Una vez hecho esto, se sigue con un reporte de calidad de los
datos \citep{Kelleher2015}. Este reporte permite tener una vista general
de los datos que permite tomar decisiones inmediatas para iniciar el
proceso de depuración de la información, hasta tener un conjunto de
datos de calidad para el análisis.

Vamos a ver un ejemplo que permita clarificar lo que hasta aquí se ha
escrito. Se cuenta con un conjunto de datos de una tienda departamental
con la que se pretende saber los factores que influyen para que el
primer cliente que llega a la tienda compre algún producto. Las
variables con las que se cuenta para este análisis son: id, edad,
genero, ocupación, estado.civil, hora.entrada, tiempo.at, vis.por.mes,
venta.

\begin{longtable}[]{@{}lllll@{}}
\caption{\label{tab:estructura-de-datos} Resumen técnico de la estructura de
los datos}\tabularnewline
\toprule
\begin{minipage}[b]{0.06\columnwidth}\raggedright\strut
Nombre\strut
\end{minipage} & \begin{minipage}[b]{0.06\columnwidth}\raggedright\strut
Tipo\strut
\end{minipage} & \begin{minipage}[b]{0.06\columnwidth}\raggedright\strut
Rol\strut
\end{minipage} & \begin{minipage}[b]{0.42\columnwidth}\raggedright\strut
Valores\strut
\end{minipage} & \begin{minipage}[b]{0.27\columnwidth}\raggedright\strut
Descripción\strut
\end{minipage}\tabularnewline
\midrule
\endfirsthead
\toprule
\begin{minipage}[b]{0.06\columnwidth}\raggedright\strut
Nombre\strut
\end{minipage} & \begin{minipage}[b]{0.06\columnwidth}\raggedright\strut
Tipo\strut
\end{minipage} & \begin{minipage}[b]{0.06\columnwidth}\raggedright\strut
Rol\strut
\end{minipage} & \begin{minipage}[b]{0.42\columnwidth}\raggedright\strut
Valores\strut
\end{minipage} & \begin{minipage}[b]{0.27\columnwidth}\raggedright\strut
Descripción\strut
\end{minipage}\tabularnewline
\midrule
\endhead
\begin{minipage}[t]{0.06\columnwidth}\raggedright\strut
ID\strut
\end{minipage} & \begin{minipage}[t]{0.06\columnwidth}\raggedright\strut
continua\strut
\end{minipage} & \begin{minipage}[t]{0.06\columnwidth}\raggedright\strut
predictiva\strut
\end{minipage} & \begin{minipage}[t]{0.42\columnwidth}\raggedright\strut
CLI\{3\}{[}0-9{]}\strut
\end{minipage} & \begin{minipage}[t]{0.27\columnwidth}\raggedright\strut
Identificador del cliente\strut
\end{minipage}\tabularnewline
\begin{minipage}[t]{0.06\columnwidth}\raggedright\strut
EDAD\strut
\end{minipage} & \begin{minipage}[t]{0.06\columnwidth}\raggedright\strut
continua\strut
\end{minipage} & \begin{minipage}[t]{0.06\columnwidth}\raggedright\strut
predictiva\strut
\end{minipage} & \begin{minipage}[t]{0.42\columnwidth}\raggedright\strut
\{*3\}{[}0-9{]}\strut
\end{minipage} & \begin{minipage}[t]{0.27\columnwidth}\raggedright\strut
Edad en años\strut
\end{minipage}\tabularnewline
\begin{minipage}[t]{0.06\columnwidth}\raggedright\strut
GENERO\strut
\end{minipage} & \begin{minipage}[t]{0.06\columnwidth}\raggedright\strut
categórica\strut
\end{minipage} & \begin{minipage}[t]{0.06\columnwidth}\raggedright\strut
predictiva\strut
\end{minipage} & \begin{minipage}[t]{0.42\columnwidth}\raggedright\strut
Femenino\strut
\end{minipage} & \begin{minipage}[t]{0.27\columnwidth}\raggedright\strut
Masculino\strut
\end{minipage}\tabularnewline
\begin{minipage}[t]{0.06\columnwidth}\raggedright\strut
OCUPACIÓN\strut
\end{minipage} & \begin{minipage}[t]{0.06\columnwidth}\raggedright\strut
categórica\strut
\end{minipage} & \begin{minipage}[t]{0.06\columnwidth}\raggedright\strut
predictiva\strut
\end{minipage} & \begin{minipage}[t]{0.42\columnwidth}\raggedright\strut
Empleado\strut
\end{minipage} & \begin{minipage}[t]{0.27\columnwidth}\raggedright\strut
estudiante\strut
\end{minipage}\tabularnewline
\begin{minipage}[t]{0.06\columnwidth}\raggedright\strut
H.ENT\strut
\end{minipage} & \begin{minipage}[t]{0.06\columnwidth}\raggedright\strut
continua\strut
\end{minipage} & \begin{minipage}[t]{0.06\columnwidth}\raggedright\strut
predictiva\strut
\end{minipage} & \begin{minipage}[t]{0.42\columnwidth}\raggedright\strut
\{*2\}{[}0-9{]}.\{*2\}{[}0-9{]}\strut
\end{minipage} & \begin{minipage}[t]{0.27\columnwidth}\raggedright\strut
Hora del día a la que entró el primer cliente a la tienda\strut
\end{minipage}\tabularnewline
\begin{minipage}[t]{0.06\columnwidth}\raggedright\strut
TI.AT\strut
\end{minipage} & \begin{minipage}[t]{0.06\columnwidth}\raggedright\strut
continua\strut
\end{minipage} & \begin{minipage}[t]{0.06\columnwidth}\raggedright\strut
predictiva\strut
\end{minipage} & \begin{minipage}[t]{0.42\columnwidth}\raggedright\strut
\{2\}{[}0-9{]}\strut
\end{minipage} & \begin{minipage}[t]{0.27\columnwidth}\raggedright\strut
Tiempo en minutos en el que se atendió al cliente\strut
\end{minipage}\tabularnewline
\begin{minipage}[t]{0.06\columnwidth}\raggedright\strut
VIS.POR.MES\strut
\end{minipage} & \begin{minipage}[t]{0.06\columnwidth}\raggedright\strut
categórica\strut
\end{minipage} & \begin{minipage}[t]{0.06\columnwidth}\raggedright\strut
predictiva\strut
\end{minipage} & \begin{minipage}[t]{0.42\columnwidth}\raggedright\strut
Nunca\strut
\end{minipage} & \begin{minipage}[t]{0.27\columnwidth}\raggedright\strut
menos de una vez por semana\strut
\end{minipage}\tabularnewline
\begin{minipage}[t]{0.06\columnwidth}\raggedright\strut
VENTA\strut
\end{minipage} & \begin{minipage}[t]{0.06\columnwidth}\raggedright\strut
categórica\strut
\end{minipage} & \begin{minipage}[t]{0.06\columnwidth}\raggedright\strut
Objetivo\strut
\end{minipage} & \begin{minipage}[t]{0.42\columnwidth}\raggedright\strut
Si\strut
\end{minipage} & \begin{minipage}[t]{0.27\columnwidth}\raggedright\strut
No\strut
\end{minipage}\tabularnewline
\bottomrule
\end{longtable}

La Tabla \ref{tab:estructura-de-datos} muestra el resumen técnico de las
variables. Se puede observar que se tiene ocho variables, siete de ellas
son variables predictivas y una es la variable dependiente o variable
objetivo. La variable objetivo es la variable de más interés en este
conjunto de datos puesto que es la variable que se desea predecir a
partir de las otras variables. También se puede observar que la variable
objetivo es de tipo categórica, llamada en muchas ocasiones dicotómica,
puesto que sólo puede tomar una de dos posibles valores (SI \textbar{}
NO). De las variables predictivas, cuatro son continuas y tres son
categóricas, en cada una se especifica los posibles valores que la
variable puede tomar. Por ejemplo, que la variable Hora.entrada puede
tomar un valor con formato \{*2\}{[}0-9{]}.\{*2\}{[}0-9{]}, se lee
\{*2\}=uno o dos números del cero al nueve {[}0-9{]}, seguido de un
punto (.), seguido de uno o dos números más \{*2\} del cero al nueve
{[}0-9{]}.

Los datos originales como fueron capturados se muestran en la tabla
\ref{tab:datos-crudos}. Al hacer una exploración rápida a la tabla, es
posible observar algunas irregularidades que pueden ser corregidas de
inmediato y otras que requieren de un proceso computacional más
elaborado, debido a que manualmente tomaría mucho tiempo hacerlo
(pensemos en que en este ejemplo son pocos datos, sin embargo, podría
tenerse una base de datos con cientos o miles de registros). La primera
irregularidad que se observa es que el registro del cliente CLI011 no
cuenta con datos, lo cual es indicativo de que este registro puede
eliminarse del conjunto de datos sin ningún problema, de hecho, es
beneficioso eliminarlo. Otro factor importante a resaltar es la captura
de los datos en H.ENT y TI.AT, en el caso de H.ENT, ningún dato está en
el formato adecuado (\{2\}{[}0-9{]}.\{2\}{[}0-9{]}), mientras que en
TI.AT algunos datos no están en el formato adecuado. Además, algunos
registros cuentan con valores faltantes.

\begin{Shaded}
\begin{Highlighting}[]
\NormalTok{dcrudos<-}\KeywordTok{read.csv}\NormalTok{(}\DataTypeTok{file=}\StringTok{"C:/Master/Libro-Pearson-CETYS/Datos-ejemplo-preproceso.csv"}\NormalTok{,}\DataTypeTok{header =} \NormalTok{T,}\DataTypeTok{sep =} \StringTok{","}\NormalTok{)}
\end{Highlighting}
\end{Shaded}

\begin{Shaded}
\begin{Highlighting}[]
\NormalTok{knitr::}\KeywordTok{kable}\NormalTok{(}
  \NormalTok{dcrudos, }\DataTypeTok{caption =} \StringTok{'Datos de los primeros clientes entrando a la tienda.'}\NormalTok{,}
  \DataTypeTok{booktabs =} \OtherTok{TRUE}
\NormalTok{)}
\end{Highlighting}
\end{Shaded}

\begin{table}

\caption{\label{tab:datos-crudos}Datos de los primeros clientes entrando a la tienda.}
\centering
\begin{tabular}[t]{lrllllll}
\toprule
ID & EDAD & GENERO & OCUPACION & H.ENT & TI.AT & VIS.POR.MES & VENTA\\
\midrule
CLI001 & 25 & Femenino & Empleado & 10:30 & 15 & Una o dos veces por semana & SI\\
CLI002 & 26 & Masculino & Empleado & 11:00 & 3 & Nunca & SI\\
CLI003 & 22 & Masculino & Empleado & 11:00 & NA & Menos de una vez por semana & NO\\
CLI004 & 41 & Masculino & Empleado & 10:30am & 15 & Una o dos veces por semana & NO\\
CLI005 & 38 & Masculino & Empleado & NA & 10 & Nunca & SI\\
\addlinespace
CLI006 & NA & Masculino & Empleado & 12:00 AM & 15 & Una o dos veces por semana & NO\\
CLI007 & 22 & Masculino & Empleado & 11:00 & 5 & Nunca & SI\\
CLI008 & 40 & Femenino & Empleado & 11:30 & 23 min & Tres o más veces por semana & NO\\
CLI009 & 34 & Masculino & Empleado & 10:00 & 12 & Una o dos veces por semana & NO\\
CLI010 & 27 & Masculino & Empleado & 10:30 & 13 & Una o dos veces por semana & NO\\
\addlinespace
CLI011 & NA & NA & NA & NA & NA & NA & NO\\
CLI012 & 52 & Femenino & Empleado & 11:00 PM & 15 & Menos de una vez por semana & SI\\
CLI013 & 48 & Femenino & Empleado & 11:50 PM & 60 min & Tres o más veces por semana & NO\\
CLI014 & 51 & Masculino & Empleado & 12:30 & 5 & Nunca & SI\\
CLI015 & 26 & Femenino & Empleado & 10:00pm & 18 & Una o dos veces por semana & NO\\
\addlinespace
CLI016 & 48 & Masculino & Empleado & 12.3 & 2 & Nunca & SI\\
CLI017 & 35 & Femenino & Empleado & 11:30 & 10 & Nunca & SI\\
CLI018 & 41 & Femenino & Empleado & 9:30 & 15 & Una o dos veces por semana & SI\\
CLI019 & 53 & Femenino & Empleado & 11:00 & NA & Menos de una vez por semana & NO\\
CLI020 & 36 & Femenino & Empleado & NA & 10 & Nunca & SI\\
CLI021 & 51 & Femenino & Empleado & 11:00 & 15 & Una o dos veces por semana & SI\\
\bottomrule
\end{tabular}
\end{table}

\begin{Shaded}
\begin{Highlighting}[]
\CommentTok{#se leimina el registro número 11 que no tiene datos}
\NormalTok{dprepro_1<-dcrudos[-}\DecValTok{11}\NormalTok{,]}
\CommentTok{#se reconstruyen los indices del conjunto de datos}
\KeywordTok{rownames}\NormalTok{(dprepro_1)<-}\DecValTok{1}\NormalTok{:}\KeywordTok{dim}\NormalTok{(dprepro_1)[}\DecValTok{1}\NormalTok{]}
\end{Highlighting}
\end{Shaded}

\begin{Shaded}
\begin{Highlighting}[]
\NormalTok{####se valida y transforman los datos de tiempo y edad###}

\CommentTok{#se cargan las funciones que se utilizará}

\KeywordTok{source}\NormalTok{(}\DataTypeTok{file=}\StringTok{"C:/Master/Libro-Pearson-CETYS/material-complementario/code/valida_edad.R"}\NormalTok{,}\DataTypeTok{encoding =} \StringTok{"UTF8"}\NormalTok{)}
\KeywordTok{source}\NormalTok{(}\DataTypeTok{file=}\StringTok{"C:/Master/Libro-Pearson-CETYS/material-complementario/code/valida_tiempo.R"}\NormalTok{,}\DataTypeTok{encoding =} \StringTok{"UTF8"}\NormalTok{)}
\KeywordTok{source}\NormalTok{(}\DataTypeTok{file=}\StringTok{"C:/Master/Libro-Pearson-CETYS/material-complementario/code/valida_minutos.R"}\NormalTok{,}\DataTypeTok{encoding =} \StringTok{"UTF8"}\NormalTok{)}
\KeywordTok{source}\NormalTok{(}\DataTypeTok{file=}\StringTok{"C:/Master/Libro-Pearson-CETYS/material-complementario/code/corrige_tiempo.R"}\NormalTok{,}\DataTypeTok{encoding =} \StringTok{"UTF8"}\NormalTok{)}
\KeywordTok{source}\NormalTok{(}\DataTypeTok{file=}\StringTok{"C:/Master/Libro-Pearson-CETYS/material-complementario/code/regeneraImputedDS.R"}\NormalTok{,}\DataTypeTok{encoding =} \StringTok{"UTF8"}\NormalTok{)}


\NormalTok{dprepro_1$H.ENT<-}\KeywordTok{sapply}\NormalTok{(dprepro_1$H.ENT,valida_tiempo)}
\NormalTok{dprepro_1$EDAD<-}\KeywordTok{sapply}\NormalTok{(dprepro_1$EDAD,valida_edad)}
\NormalTok{dprepro_1$TI.AT<-}\KeywordTok{sapply}\NormalTok{(dprepro_1$TI.AT,valida_minutos)}
\NormalTok{dprepro_1$EDAD<-}\KeywordTok{as.numeric}\NormalTok{(dprepro_1$EDAD)}
\NormalTok{dprepro_1$TI.AT<-}\KeywordTok{as.numeric}\NormalTok{(dprepro_1$TI.AT)}
\end{Highlighting}
\end{Shaded}

Después de estos cambios el conjunto de datos luce de la siguiente forma

\begin{Shaded}
\begin{Highlighting}[]
\NormalTok{knitr::}\KeywordTok{kable}\NormalTok{(}
  \NormalTok{dprepro_1, }\DataTypeTok{caption =} \StringTok{'Datos después de la validación y transformación'}\NormalTok{,}
  \DataTypeTok{booktabs =} \OtherTok{TRUE}
\NormalTok{)}
\end{Highlighting}
\end{Shaded}

\begin{table}

\caption{\label{tab:datos-despues-valitrans}Datos después de la validación y transformación}
\centering
\begin{tabular}[t]{lrllrrll}
\toprule
ID & EDAD & GENERO & OCUPACION & H.ENT & TI.AT & VIS.POR.MES & VENTA\\
\midrule
CLI001 & 25 & Femenino & Empleado & 10.50 & 15 & Una o dos veces por semana & SI\\
CLI002 & 26 & Masculino & Empleado & 11.00 & 3 & Nunca & SI\\
CLI003 & 22 & Masculino & Empleado & 11.00 & NA & Menos de una vez por semana & NO\\
CLI004 & 41 & Masculino & Empleado & 10.50 & 15 & Una o dos veces por semana & NO\\
CLI005 & 38 & Masculino & Empleado & NA & 10 & Nunca & SI\\
\addlinespace
CLI006 & NA & Masculino & Empleado & 12.00 & 15 & Una o dos veces por semana & NO\\
CLI007 & 22 & Masculino & Empleado & 11.00 & 5 & Nunca & SI\\
CLI008 & 40 & Femenino & Empleado & 11.50 & 23 & Tres o más veces por semana & NO\\
CLI009 & 34 & Masculino & Empleado & 10.00 & 12 & Una o dos veces por semana & NO\\
CLI010 & 27 & Masculino & Empleado & 10.50 & 13 & Una o dos veces por semana & NO\\
\addlinespace
CLI012 & 52 & Femenino & Empleado & 11.00 & 15 & Menos de una vez por semana & SI\\
CLI013 & 48 & Femenino & Empleado & 11.83 & 60 & Tres o más veces por semana & NO\\
CLI014 & 51 & Masculino & Empleado & 12.50 & 5 & Nunca & SI\\
CLI015 & 26 & Femenino & Empleado & 10.00 & 18 & Una o dos veces por semana & NO\\
CLI016 & 48 & Masculino & Empleado & 12.05 & 2 & Nunca & SI\\
\addlinespace
CLI017 & 35 & Femenino & Empleado & 11.50 & 10 & Nunca & SI\\
CLI018 & 41 & Femenino & Empleado & 9.50 & 15 & Una o dos veces por semana & SI\\
CLI019 & 53 & Femenino & Empleado & 11.00 & NA & Menos de una vez por semana & NO\\
CLI020 & 36 & Femenino & Empleado & NA & 10 & Nunca & SI\\
CLI021 & 51 & Femenino & Empleado & 11.00 & 15 & Una o dos veces por semana & SI\\
\bottomrule
\end{tabular}
\end{table}

Al hacer el reporte de calidad del conjunto de datos se podrán ver estas
irregularidades y otras de las que no nos percatamos a simple vista. Es
importante notar que en este ejemplo, se tienen pocos datos, sin embargo
en un escenario real se tendrían tantos datos que las irregularidades
advertidas de forma simple, podrían fácilmente pasarse por alto.

\begin{Shaded}
\begin{Highlighting}[]
\NormalTok{VARCONT<-}\KeywordTok{QOfContinuousF}\NormalTok{(}\KeywordTok{cbind.data.frame}\NormalTok{(}\DataTypeTok{EDAD=}\NormalTok{dprepro_1$EDAD,}\DataTypeTok{H.ENT=}\NormalTok{dprepro_1$H.ENT,}\DataTypeTok{TI.AT=}\NormalTok{dprepro_1$TI.AT))}
\NormalTok{VARCAT<-}\KeywordTok{QOfCategoricalF}\NormalTok{(}\KeywordTok{cbind.data.frame}\NormalTok{(}\DataTypeTok{ID=}\NormalTok{dprepro_1$ID,dprepro_1[,}\DecValTok{3}\NormalTok{:}\DecValTok{4}\NormalTok{],dprepro_1[,}\DecValTok{7}\NormalTok{:}\DecValTok{8}\NormalTok{]))}
\end{Highlighting}
\end{Shaded}

\begin{Shaded}
\begin{Highlighting}[]
\NormalTok{knitr::}\KeywordTok{kable}\NormalTok{(}
  \NormalTok{VARCONT, }\DataTypeTok{caption =} \StringTok{'Reporte de calidad de datos: Variables continuas'}\NormalTok{,}
  \DataTypeTok{booktabs =} \OtherTok{FALSE}\NormalTok{)}
\end{Highlighting}
\end{Shaded}

\begin{table}

\caption{\label{tab:reporte-de-calidad-varcont}Reporte de calidad de datos: Variables continuas}
\centering
\begin{tabular}[t]{l|r|r|r|r|r|r|r|r|r|r}
\hline
  & Count & Miss & Card & Min & Qrt1 & Median & Qrt3 & Max & Mean & Sdev\\
\hline
EDAD & 20 & 1 & 14 & 22.0 & 26.5 & 38 & 48.0 & 53.0 & 37.68 & 10.80\\
\hline
H.ENT & 20 & 2 & 9 & 9.5 & 10.5 & 11 & 11.5 & 12.5 & 11.02 & 0.79\\
\hline
TI.AT & 20 & 2 & 10 & 2.0 & 10.0 & 14 & 15.0 & 60.0 & 14.50 & 12.59\\
\hline
\end{tabular}
\end{table}

\begin{Shaded}
\begin{Highlighting}[]
\CommentTok{#write.csv(file = "C:/Master/Libro-Pearson-CETYS/material-complementario/tables/repcontvar.csv",VARCONT)}
\end{Highlighting}
\end{Shaded}

En la Tabla \ref{tab:reporte-de-calidad-varcont}, se pueden observar
algunos detalles a considerar para hacer tratamiento de los datos antes
de que estén listos para ser guardados en la base de conocimientos. La
columna de datos faltantes muestra que existen cinco registros que no
tiene valores en las variables continuas (1 en edad, 2 en hora de
entrada y 2 en tiempo de atención). Se puede optar por eliminar los
registros con datos faltantes o bien, utilizar una técnica de
imputación\footnote{La imputación es un proceso que permite suplir los
  datos faltantes con datos aproximados a los valores reales que se
  calculan por métodos estadísticos y/o matemáticos a partir de los
  existentes.}. También puede observarse que en el tiempo de atención
existe una desviación estándar bastante amplia (15.18). Tomando en
cuenta que la media es de 12.59, el valor mínimo 2 y el valor máximo 60,
puede anticiparse la posibilidad de valores atípicos, es decir valores
que son excepcionales y que muchas veces no conviene tomarlos en cuenta
para los análisis predictivos porque ocurren con muy poca frecuencia en
la realidad y pueden ocasionar que los modelos no se ajusten a los datos
de forma debida. En resumen, se debe dar tratamiento a los datos
faltantes y se debe buscar si existen valores atípicos para dejarlos
fuera de la generación de los modelos\footnote{Cabe aclarar que estos
  datos no se eliminan completamente de las bases de datos, es útil
  reportarlos como existentes porque proveen información valiosa para
  los analistas, sin embargo, si se dejan fuera de muchos procesos para
  evitar sesgos o como se dijo anteriormente, para evitar modelos
  predictivos menos eficientes.}.

\begin{Shaded}
\begin{Highlighting}[]
\NormalTok{knitr::}\KeywordTok{kable}\NormalTok{(}
  \NormalTok{VARCAT, }\DataTypeTok{caption =} \StringTok{'Reporte de calidad de datos: Variables continuas'}\NormalTok{,}
  \DataTypeTok{booktabs =} \OtherTok{FALSE}\NormalTok{)}
\end{Highlighting}
\end{Shaded}

\begin{table}

\caption{\label{tab:reporte-de-calidad-varcat}Reporte de calidad de datos: Variables continuas}
\centering
\begin{tabular}[t]{l|r|r|r|l|r|l|l|r|l}
\hline
  & Count & Miss & Card & Mode & ModeFrec & ModePerc & Mode2 & Mode2Frec & Mode2Perc\\
\hline
ID & 20 & 0 & 20 & CLI001 & 1 & 5\% & CLI002 & 1 & 5\%\\
\hline
GENERO & 20 & 0 & 2 & Femenino & 10 & 50\% & Masculino & 10 & 50\%\\
\hline
OCUPACION & 20 & 0 & 1 & Empleado & 20 & 100\% & NA & NA & NA\%\\
\hline
VIS.POR.MES & 20 & 0 & 4 & Una o dos veces por semana & 8 & 40\% & Nunca & 7 & 35\%\\
\hline
VENTA & 20 & 0 & 2 & SI & 11 & 55\% & NO & 9 & 45\%\\
\hline
\end{tabular}
\end{table}

\begin{Shaded}
\begin{Highlighting}[]
\CommentTok{#write.csv(file = "C:/Master/Libro-Pearson-CETYS/material-complementario/tables/repcatvar.csv",VARCAT)}
\end{Highlighting}
\end{Shaded}

En el reporte de calidad de las variables categóricas mostrado en la
Tabla \ref{tab:reporte-de-calidad-varcat}, no existen datos faltantes,
sin embargo, la columna ``cardinalidad'', aporta información relevante
que debe tomarse en consideración. En primer lugar, la variable ID tiene
cardinalidad 20, significa que existen 20 ID distintos en 20 registros.
La tienda fue visitada por 20 clientes distintos, de esta forma se evita
pensar que algunos resultados de VENTA/NO VENTA se debe a un determinado
cliente (BUEN COMPRADOR/MAL COMPRADOR), así que, esta columna podría
eliminarse del conjunto de datos puesto que no aporta información
relevante al análisis. Se tiene el caso contrario, la cardinalidad de la
variable OCUPACIÓN, tiene cardinalidad uno, es decir todos los registros
tienen el mismo valor en esta variable (``Empleado''). Esta columna
también puede eliminarse debido a que la VENTA/NO VENTA en este caso no
tiene que ver con la ocupación de las personas. Las demás variables se
observan con buena calidad y serán utilizadas para el análisis.

Con la examinación de los datos en las tablas
\ref{tab:reporte-de-calidad-varcont} y
\ref{tab:reporte-de-calidad-varcat}, se genera el plan para atender las
irregularidades en la tabla \ref{tab:plan-de-atencion-irregularidades}

\begin{longtable}[]{@{}lll@{}}
\caption{\label{tab:plan-de-atencion-irregularidades} Plan para atención de
irregularidades en los datos}\tabularnewline
\toprule
\begin{minipage}[b]{0.05\columnwidth}\raggedright\strut
Variable\strut
\end{minipage} & \begin{minipage}[b]{0.11\columnwidth}\raggedright\strut
Irregularidad\strut
\end{minipage} & \begin{minipage}[b]{0.75\columnwidth}\raggedright\strut
Plan\strut
\end{minipage}\tabularnewline
\midrule
\endfirsthead
\toprule
\begin{minipage}[b]{0.05\columnwidth}\raggedright\strut
Variable\strut
\end{minipage} & \begin{minipage}[b]{0.11\columnwidth}\raggedright\strut
Irregularidad\strut
\end{minipage} & \begin{minipage}[b]{0.75\columnwidth}\raggedright\strut
Plan\strut
\end{minipage}\tabularnewline
\midrule
\endhead
\begin{minipage}[t]{0.05\columnwidth}\raggedright\strut
EDAD, H.ENT, TI.AT\strut
\end{minipage} & \begin{minipage}[t]{0.11\columnwidth}\raggedright\strut
No todos los datos están en formato correcto\strut
\end{minipage} & \begin{minipage}[t]{0.75\columnwidth}\raggedright\strut
En el caso de la edad existen registros con la palabra ``años''
agregada, en H.ENT los datos están separados por (:) en lugar de (.),
estos datos serán convertidos al nuevo formato (por ejemplo, 11:30
pasará a 11.5 en formato decimal). La variable TI.AT será validada y
transformada según el formato especificado.\strut
\end{minipage}\tabularnewline
\begin{minipage}[t]{0.05\columnwidth}\raggedright\strut
EDAD, H.ENT, TI.AT\strut
\end{minipage} & \begin{minipage}[t]{0.11\columnwidth}\raggedright\strut
Valores faltantes\strut
\end{minipage} & \begin{minipage}[t]{0.75\columnwidth}\raggedright\strut
Se utilizará un método de imputación para suplir los valores
faltantes\strut
\end{minipage}\tabularnewline
\begin{minipage}[t]{0.05\columnwidth}\raggedright\strut
TI.AT\strut
\end{minipage} & \begin{minipage}[t]{0.11\columnwidth}\raggedright\strut
Desviación estándar amplia\strut
\end{minipage} & \begin{minipage}[t]{0.75\columnwidth}\raggedright\strut
Se utilizará un método para identificar valores atípicos, en el caso de
existir, los datos serán excluidos del análisis.\strut
\end{minipage}\tabularnewline
\begin{minipage}[t]{0.05\columnwidth}\raggedright\strut
ID\strut
\end{minipage} & \begin{minipage}[t]{0.11\columnwidth}\raggedright\strut
Cardinalidad muy alta\strut
\end{minipage} & \begin{minipage}[t]{0.75\columnwidth}\raggedright\strut
La columna será excluida de los datos para el análisis, no aporta
información relevante.\strut
\end{minipage}\tabularnewline
\begin{minipage}[t]{0.05\columnwidth}\raggedright\strut
OCUPACION\strut
\end{minipage} & \begin{minipage}[t]{0.11\columnwidth}\raggedright\strut
Cardinalidad muy baja\strut
\end{minipage} & \begin{minipage}[t]{0.75\columnwidth}\raggedright\strut
La columna será excluida de los datos para el análisis, no aporta
información relevante.\strut
\end{minipage}\tabularnewline
\bottomrule
\end{longtable}

El siguiente paso en la depuración de la base de datos es la
identificación de valores atípicos que serán descartados del conjunto de
datos que servirá para la obtención de conjeturas o la generación
modelos que permitan hacer pronósticos para beneficio de la empresa.

\begin{Shaded}
\begin{Highlighting}[]
\KeywordTok{library}\NormalTok{(outliers)}
\KeywordTok{library}\NormalTok{(magrittr)}

\CommentTok{#se descartan los registros con valores faltantes en la variable TIEMPO.AT para el}
\CommentTok{#analisis de los valores atípicos}
\NormalTok{nasdp<-}\KeywordTok{which}\NormalTok{(}\KeywordTok{is.na}\NormalTok{(dprepro_1$TI.AT))}
\NormalTok{dptemp<-dprepro_1[-nasdp,]}
\KeywordTok{rownames}\NormalTok{(dptemp)<-}\DecValTok{1}\NormalTok{:}\KeywordTok{nrow}\NormalTok{(dptemp) }

\NormalTok{ic<-}\FloatTok{0.997}
\CommentTok{#hist(dprepro_1$TIEMPO.AT)}
\NormalTok{outTAT<-}\KeywordTok{scores}\NormalTok{(dptemp$TI.AT,}\DataTypeTok{type =} \StringTok{"z"}\NormalTok{,}\DataTypeTok{prob =} \NormalTok{ic)%>%}\KeywordTok{which}\NormalTok{(.)}
\NormalTok{outTAT<-}\KeywordTok{unique}\NormalTok{(outTAT)}

\NormalTok{datipico<-dptemp[outTAT,]}

\CommentTok{#elimina los registros donde hay valores atípicos}
\NormalTok{dptemp<-dptemp[-outTAT,]}
\CommentTok{#reestructura los índices del dataset}
\KeywordTok{rownames}\NormalTok{(dptemp)<-}\DecValTok{1}\NormalTok{:}\KeywordTok{nrow}\NormalTok{(dptemp)}

\CommentTok{#se restructuran los registros con valores faltantes en d_preproc1 descartando el registo con valor atípico}
\NormalTok{dprepro_1<-}\KeywordTok{rbind.data.frame}\NormalTok{(dprepro_1[nasdp,],dptemp)}
\KeywordTok{rownames}\NormalTok{(dprepro_1)<-}\DecValTok{1}\NormalTok{:}\KeywordTok{nrow}\NormalTok{(dprepro_1)}

\NormalTok{dordered<-dprepro_1[}\KeywordTok{order}\NormalTok{(dprepro_1$ID),]}
\end{Highlighting}
\end{Shaded}

Existe un registro con un dato atípico en la variable TIEMPO.AT (ver
Tabla \ref{tab:dato-atipico}), este registro se encuentra por arriba de
las tres desviaciones estándar, así que será excluido de los datos que
serán considerados para análisis posteriores.

\begin{Shaded}
\begin{Highlighting}[]
\NormalTok{knitr::}\KeywordTok{kable}\NormalTok{(datipico,}\DataTypeTok{caption =} \StringTok{'Dato atípico identificado'}\NormalTok{,}\DataTypeTok{booktabs =} \OtherTok{FALSE}\NormalTok{) }
\end{Highlighting}
\end{Shaded}

\begin{table}

\caption{\label{tab:dato-atipico}Dato atípico identificado}
\centering
\begin{tabular}[t]{l|l|r|l|l|r|r|l|l}
\hline
  & ID & EDAD & GENERO & OCUPACION & H.ENT & TI.AT & VIS.POR.MES & VENTA\\
\hline
11 & CLI013 & 48 & Femenino & Empleado & 11.83 & 60 & Tres o más veces por semana & NO\\
\hline
\end{tabular}
\end{table}

Como penúltimo paso, se inputarán los valores faltantes mediante
regresión logítica multivariable proporcionada por el paquete
\emph{mice} \citep{mice2011}.

\begin{Shaded}
\begin{Highlighting}[]
\KeywordTok{library}\NormalTok{(}\StringTok{"mice"}\NormalTok{)}
\end{Highlighting}
\end{Shaded}

\begin{verbatim}
## 
## Attaching package: 'mice'
\end{verbatim}

\begin{verbatim}
## The following object is masked from 'package:RCurl':
## 
##     complete
\end{verbatim}

\begin{Shaded}
\begin{Highlighting}[]
  \CommentTok{#busca las columnas que tienen valores faltantes}
  \NormalTok{colNA<-(}\KeywordTok{c}\NormalTok{(}\KeywordTok{which}\NormalTok{(}\KeywordTok{is.na}\NormalTok{(dordered)))%/%}\KeywordTok{dim}\NormalTok{(dordered)[}\DecValTok{1}\NormalTok{])+}\DecValTok{1}
  
  \CommentTok{#elimina los índices repetidos}
  \NormalTok{colNA<-}\KeywordTok{unique}\NormalTok{(colNA)}
  \CommentTok{#construye el dataset con las columnas que tienen datos faltantes para hacer la imputacion}
  \NormalTok{qs_dataforimp<-dordered[,colNA]}
\end{Highlighting}
\end{Shaded}

\begin{Shaded}
\begin{Highlighting}[]
  \CommentTok{#imputa los datos con el dataset #dos en este caso elegido arbitrariamente  }
  \NormalTok{completeSH<-}\KeywordTok{complete}\NormalTok{(imputedSH,}\DecValTok{2}\NormalTok{)}
  
  \CommentTok{#dpreproc1_imputed<-regeneraImputedDS(dordered,completeSH,colNA,colnames(dprepro_1))}
  \NormalTok{dpreproc1_imputed<-}\KeywordTok{regeneraImputedDS}\NormalTok{(dordered,completeSH,colNA)}
\end{Highlighting}
\end{Shaded}

Las Tabla \ref{tab:faltantes-imputados}, muestra dos tablas con las
columnas que tiene los datos faltantes y las columnas que tienen los
datos imputados.

\begin{Shaded}
\begin{Highlighting}[]
\NormalTok{variablesconNa<-dprepro_1[,}\KeywordTok{c}\NormalTok{(}\DecValTok{2}\NormalTok{,}\DecValTok{5}\NormalTok{,}\DecValTok{6}\NormalTok{)]}
\NormalTok{conNA<-variablesconNa[}\KeywordTok{order}\NormalTok{(}\KeywordTok{rownames}\NormalTok{(variablesconNa)),]}
\NormalTok{sinNA<-completeSH[}\KeywordTok{order}\NormalTok{(}\KeywordTok{rownames}\NormalTok{(completeSH)),]}

\NormalTok{knitr::}\KeywordTok{kable}\NormalTok{(}
  \KeywordTok{list}\NormalTok{(}
    \NormalTok{conNA,sinNA}
  \NormalTok{),}
  \DataTypeTok{caption =} \StringTok{'Variables con datos faltantes y sus respectivas imputaciones'}\NormalTok{,}\DataTypeTok{booktabs=}\OtherTok{TRUE}
\NormalTok{)}
\end{Highlighting}
\end{Shaded}

\begin{table}
\caption{\label{tab:faltantes-imputados}Variables con datos faltantes y sus respectivas imputaciones}

\centering
\begin{tabular}[t]{lrrr}
\toprule
  & EDAD & H.ENT & TI.AT\\
\midrule
1 & 22 & 11.00 & NA\\
10 & 34 & 10.00 & 12\\
11 & 27 & 10.50 & 13\\
12 & 52 & 11.00 & 15\\
13 & 51 & 12.50 & 5\\
\addlinespace
14 & 26 & 10.00 & 18\\
15 & 48 & 12.05 & 2\\
16 & 35 & 11.50 & 10\\
17 & 41 & 9.50 & 15\\
18 & 36 & NA & 10\\
\addlinespace
19 & 51 & 11.00 & 15\\
2 & 53 & 11.00 & NA\\
3 & 25 & 10.50 & 15\\
4 & 26 & 11.00 & 3\\
5 & 41 & 10.50 & 15\\
\addlinespace
6 & 38 & NA & 10\\
7 & NA & 12.00 & 15\\
8 & 22 & 11.00 & 5\\
9 & 40 & 11.50 & 23\\
\bottomrule
\end{tabular}
\centering
\begin{tabular}[t]{lrrr}
\toprule
  & EDAD & H.ENT & TI.AT\\
\midrule
1 & 22 & 11.00 & 2\\
10 & 34 & 10.00 & 12\\
11 & 27 & 10.50 & 13\\
12 & 52 & 11.00 & 15\\
13 & 51 & 12.50 & 5\\
\addlinespace
14 & 26 & 10.00 & 18\\
15 & 48 & 12.05 & 2\\
16 & 35 & 11.50 & 10\\
17 & 41 & 9.50 & 15\\
18 & 36 & 11.00 & 10\\
\addlinespace
19 & 51 & 11.00 & 15\\
2 & 53 & 11.00 & 12\\
3 & 25 & 10.50 & 15\\
4 & 26 & 11.00 & 3\\
5 & 41 & 10.50 & 15\\
\addlinespace
6 & 38 & 11.00 & 10\\
7 & 40 & 12.00 & 15\\
8 & 22 & 11.00 & 5\\
9 & 40 & 11.50 & 23\\
\bottomrule
\end{tabular}
\end{table}

Como último paso, se eliminan las columnas con datos irrelevantes por su
cardinalidad como se explicó anteriormente.

\begin{Shaded}
\begin{Highlighting}[]
\NormalTok{dpreproc1_imputed<-dpreproc1_imputed[,-}\KeywordTok{c}\NormalTok{(}\DecValTok{1}\NormalTok{,}\DecValTok{4}\NormalTok{)]}
\CommentTok{#write.csv(file = "C:/Master/Libro-Pearson-CETYS/material-complementario/tables/datos-listos.csv",dpreproc1_imputed)}
\end{Highlighting}
\end{Shaded}

\begin{Shaded}
\begin{Highlighting}[]
\NormalTok{knitr::}\KeywordTok{kable}\NormalTok{(}
  \NormalTok{dpreproc1_imputed,}
  \DataTypeTok{caption =} \StringTok{'Conjunto de datos preparado para análisis'}\NormalTok{,}\DataTypeTok{booktabs=}\OtherTok{TRUE}
\NormalTok{)}
\end{Highlighting}
\end{Shaded}

\begin{table}

\caption{\label{tab:datos-listos}Conjunto de datos preparado para análisis}
\centering
\begin{tabular}[t]{lrlrrll}
\toprule
  & EDAD & GENERO & H.ENT & TI.AT & VIS.POR.MES & VENTA\\
\midrule
3 & 25 & Femenino & 10.50 & 15 & Una o dos veces por semana & SI\\
4 & 26 & Masculino & 11.00 & 3 & Nunca & SI\\
1 & 22 & Masculino & 11.00 & 2 & Menos de una vez por semana & NO\\
5 & 41 & Masculino & 10.50 & 15 & Una o dos veces por semana & NO\\
6 & 38 & Masculino & 11.00 & 10 & Nunca & SI\\
\addlinespace
7 & 40 & Masculino & 12.00 & 15 & Una o dos veces por semana & NO\\
8 & 22 & Masculino & 11.00 & 5 & Nunca & SI\\
9 & 40 & Femenino & 11.50 & 23 & Tres o más veces por semana & NO\\
10 & 34 & Masculino & 10.00 & 12 & Una o dos veces por semana & NO\\
11 & 27 & Masculino & 10.50 & 13 & Una o dos veces por semana & NO\\
\addlinespace
12 & 52 & Femenino & 11.00 & 15 & Menos de una vez por semana & SI\\
13 & 51 & Masculino & 12.50 & 5 & Nunca & SI\\
14 & 26 & Femenino & 10.00 & 18 & Una o dos veces por semana & NO\\
15 & 48 & Masculino & 12.05 & 2 & Nunca & SI\\
16 & 35 & Femenino & 11.50 & 10 & Nunca & SI\\
\addlinespace
17 & 41 & Femenino & 9.50 & 15 & Una o dos veces por semana & SI\\
2 & 53 & Femenino & 11.00 & 12 & Menos de una vez por semana & NO\\
18 & 36 & Femenino & 11.00 & 10 & Nunca & SI\\
19 & 51 & Femenino & 11.00 & 15 & Una o dos veces por semana & SI\\
\bottomrule
\end{tabular}
\end{table}

En el conjunto de datos de la Tabla \ref{tab:datos-listos}, se han
excluido las variables ID y OCUPACION, se han sustituido los valores
faltantes por nuevos valores haciendo uso del método de regresión
logística multi-variable mediante el uso del software ``mice''
\citep{R-mice}. Se eliminó un registro que contenía un dato atípico que
se identificó fuera de tres desviaciones estándar de la media, mediante
el software ``outliers'' \citep{R-outliers}. Los datos de las variables
EDAD, H.ENT y TI.AT han sido validados y transformados a valores válidos
para el proceso de análisis. El listado completo de paquetes que se han
ocupado para realizar este ejercicio incluye: \citep[\citet{R-magrittr},
\citet{R-RCurl}, \citet{R-bookdown} and \citet{R-rmarkdown}]{R-knitr}

\bibliography{packages,book}


\end{document}
